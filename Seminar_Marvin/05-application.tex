%!TEX root = vorlage.tex
\section{Application in Medical Informatics}\label{sec:application}




This section is structured as fellows: \Cref{sec:otherapp} described gives a brief overview where \gls{CNN} based semantic segmentation is already used. \Cref{sec:surgery} is a case study giving a detailed description fcn can be applied to a common task in Medical Informatics.

\subsection{Applications in Literature} \label{sec:otherapp}

Deep neural networks have been used in several recent publications in medical informatics and biotechnological science \cite{Brain_Tumor} \cite{Colon} \cite{Micro1} \cite{Wound} \cite{Micro2}. \cite{Wound} is using a decoder-encoder architecture similar to \cite{segnet} in order to automatical segment and analyse open wounds. 


\subsection{FCNs for computed aided surgery} \label{sec:surgery}

\begin{figure*}
    \begin{subfigure}[t]{0.475\columnwidth}
    \centering
        \includegraphics[width=\columnwidth]{images/img_19_raw.png}
        \caption{Bilddaten einer Operation.}
        \label{fig:sfig1}
    \end{subfigure}\hspace{0.05\textwidth}
    \begin{subfigure}[t]{0.475\columnwidth}
        \centering
        \includegraphics[width=\columnwidth]{images/img_19_class.png}
        \caption{Label der Operation. Die in Grau gezeichneten Instrumenten
                 sollen erkannt werden.}
        \label{fig:sfig2}
    \end{subfigure}
    \caption{Medizininformatik: Visuelle Erkennung von chirurgischen
             Instrumenten zur Verbesserung von chirurgischen
             Assistenzsystemen.}
    \label{fig:medInfo}
\end{figure*}

Training FCNs on this data involves two main steps:

\begin{enumerate}
\item Train a Classification Network
\item 
\end{enumerate}


