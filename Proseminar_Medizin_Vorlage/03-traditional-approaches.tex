%!TEX root = vorlage.tex
% Martin Thoma
\section{Traditional Approaches}\label{sec:traditional-approaches}
Image segmentation algorithms which use traditional approaches, hence don't
apply neural networks and make heavy use of domain knowledge, are wide-spread
in the computer vision community. This section discribes the most popular ones
and discusses their respective advantages in
\cref{subsec:traditional-approaches-discussion}.

This will not cover non-semantic segmentation algorithms. In particular, it
will not cover any algorithm which does not make use of labeled data such as
the Watershed segmentation~\cite{beucher1992morphological} and clustering
algorithms such as $k$-means~\cite{hartigan1975clustering} or mean shift
clustering~\cite{comaniciu2002mean}. However, it should be noted that such
non-semantic segmentation algorithms can be used to build semantic segmentation
algorithms.

Semantic segmentation algorithms store information about the classes they were
trained to segment.


\subsection{Graph Based Image Segmentation}\label{subsec:graph-based-image-segmentation}
\href{http://cs.brown.edu/~pff/segment/}{cs.brown.edu}
See \cite{felzenszwalb2004efficient}.

TODO: See also \cite{pantofaru2005comparison} for another description.

EDISON is a publicly available software for mean shift segmentation.\cite{christoudias2002synergism}


\subsection{Random Walks}
NOT supervised hence not semantic?!?

Notes from \cite{meilpa2001learning}:

\begin{itemize}
    \item Normalized Cut (see \cite{shi2000normalized}) seems to be a special
          case of this
    \item Seems to be a \enquote{spectral method}
    \item This is pairwise clustering. Clusters points which are similar and
          optimize e.g. for maximum total intracluster similarity. In contrast,
          to statistical clustering which assumes a probabilistic model which
          generates the data, this only defines a similarity function.
    \item Spectral clustering is a similarity based method. Spectral clustering
          methods use eingenvalues / eigenvectors of the matrix obtained by the
          similarity function.
\end{itemize}


\subsection{Markov Random Fields}
From \cite{yang2012layered}:

> In contrast, semantic segmentation models have largely been built on top of
Markov Random Field (MRF) models which enforce smoothness across pixel labels

\begin{itemize}
    \item X. He, R. Zemel, and M. Carreira-Perpinan, “Multiscale Condi-
          tional Random Fields for Image Labeling,” Proc. IEEE CS Conf.
          Computer Vision and Pattern Recognition, vol. 2, 2004.
    \item A. Torralba, K. Murphy, and W. Freeman, “Contextual Models for
          Object Detection Using Boosted Random Fields,” Proc. Advances in
          Neural Information Processing Systems, 2004.
    \item S. Kumar and M. Hebert, “A Hierarchical Field Framework for
          Unified Context-Based Classification,” Proc. 10th IEEE Int’l Conf.
          Computer Vision, vol. 2, 2005.
    \item \Glspl{CRF} were applied in \cite{shotton2006textonboost}.
    \item Z. Tu, “Auto-Context and Its Application to High-Level Vision
          Tasks,” Proc. IEEE Conf. Computer Vision and Pattern Recognition,
          2008.
\end{itemize}


\subsection{Pre-processing}\label{subsec:preprocessing}
A typical image is opened in RGB color space, but depending on the classifier
and the problem another color space might result in better segmentations. RGB,
YcBcr, HSL, Lab and YIQ are some examples used by \cite{cohen2015memory}.

\Gls{PCA} is applied by~\cite{chen2011pixel} to reduce the dimensionality of
the feature space.

\subsection{Post-processing methods}\label{subsec:post-processing-methods}
Post-processing methods are an important part of traditional pixel-level
segmentation approaches. Active contour models are one example of a
post-processing method~\cite{kass1988snakes}.


\subsection{Discussion}\label{subsec:traditional-approaches-discussion}
According to \cite{pantofaru2005comparison}, the mean shift algorithm produces
segmentations that correspond well to human perception, but it is sensitive to
its parameters. Depending on them, different granularities of the segmentation
can be achieved. \cite{pantofaru2005comparison} draws the conclusion, that
the segmentations found by the graph based image segmentation approach
in \cite{felzenszwalb2004efficient} are inferior to the segmentations found
by the mean shift algorithm described in \cite{comaniciu2002mean}.
