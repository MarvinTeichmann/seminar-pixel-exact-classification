%!TEX root = vorlage.tex
% Martin Thoma
\section{Traditional Approaches}\label{sec:traditional-approaches}

Image segmentation algorithms which use traditional approaches, hence don't
apply neural networks and make heavy use of domain knowledge, are wide-spread
in the computer vision community. This section discribes the most popular ones
and discusses their respective advantages in
\cref{subsec:traditional-approaches-discussion}.

\subsection{$k$-Means}\label{subsec:k-means}

\begin{itemize}
    \item Represent every pixel of the image as a vector (x, y, r, g, b, h, s,
          v, ...).
    \item Define distance function which weights the components
\end{itemize}


\subsection{Watershed Algorithm}\label{subsec:watershed}

\begin{itemize}
    \item Apply to image intensity gives some long superpixels
    \item apply to image gradient magnitude gives rounder superpixels
\end{itemize}

% http://docs.opencv.org/master/d3/db4/tutorial_py_watershed.html#gsc.tab=0
% http://docs.opencv.org/master/d2/dbd/tutorial_distance_transform.html#gsc.tab=0


\subsection{Mean Shift Algorithm}\label{subsec:mean-shift}
The mean shift algorithm was introduced by~\cite{comaniciu2002mean} for
segmentation tasks. The algorithm first applies mean shift filtering on the
original image data and then clusters the remaining points.

TODO: Understand this algorithm. Use \cite{comaniciu2002mean} and
\cite{pantofaru2005comparison} for it.


\subsection{Graph Based Image Segmentation}\label{subsec:graph-based-image-segmentation}
\href{http://cs.brown.edu/~pff/segment/}{cs.brown.edu}
See \cite{felzenszwalb2004efficient}.

TODO: See also \cite{pantofaru2005comparison} for another description.

EDISON is a publicly available software for mean shift segmentation.\cite{christoudias2002synergism}


\subsection{Pre-processing}\label{subsec:preprocessing}
A typical image is opened in RGB color space, but depending on the classifier
and the problem another color space might result in better segmentations. RGB,
YcBcr, HSL, Lab and YIQ are some examples used by \cite{cohen2015memory}.

\subsection{Post-processing methods}\label{subsec:post-processing-methods}
Post-processing methods are an important part of traditional pixel-level
segmentation approaches. Active contour models are one example of a
post-processing method~\cite{kass1988snakes}.


\subsection{Discussion}\label{subsec:traditional-approaches-discussion}
According to \cite{pantofaru2005comparison}, the mean shift algorithm produces
segmentations that correspond well to human perception, but it is sensitive to
its parameters. Depending on them, different granularities of the segmentation
can be achieved. \cite{pantofaru2005comparison} draws the conclusion, that
the segmentations found by the graph based image segmentation approach
in \cite{felzenszwalb2004efficient} are inferior to the segmentations found
by the mean shift algorithm described in \cite{comaniciu2002mean}.
