%!TEX root = vorlage.tex
% Martin Thoma
\section{Traditional Approaches}\label{sec:traditional-approaches}

\subsection{k-Means}\label{subsec:k-means}

\begin{itemize}
    \item Represent every pixel of the image as a vector (x, y, r, g, b, h, s,
          v, ...).
    \item Define distance function which weights the components
\end{itemize}


\subsection{Watershed Algorithm}\label{subsec:watershed}

\begin{itemize}
    \item Apply to image intensity gives some long superpixels
    \item apply to image gradient magnitude gives rounder superpixels
\end{itemize}


\subsection{Mean Shift Algorithm}\label{subsec:mean-shift}
The mean shift algorithm was introduced by~\cite{comaniciu2002mean} for
segmentation tasks.


\subsection{Graph Based Image Segmentation}\label{subsec:graph-based-image-segmentation}
\href{http://cs.brown.edu/~pff/segment/}{cs.brown.edu}
See \cite{felzenszwalb2004efficient}.
