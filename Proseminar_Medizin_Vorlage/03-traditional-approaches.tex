%!TEX root = vorlage.tex
% Martin Thoma

% definitions
\newcommand\independent{\protect\mathpalette{\protect\independenT}{\perp}}
\def\independenT#1#2{\mathrel{\rlap{$#1#2$}\mkern2mu{#1#2}}}

\section{Traditional Approaches}\label{sec:traditional-approaches}%
Image segmentation algorithms which use traditional approaches, hence don't
apply neural networks and make heavy use of domain knowledge, are wide-spread
in the computer vision community. Features which can be used for segmentation
are described in \cref{subsec:features}, a very brief overview of unsupervised,
non-semantic segmentation is given in
\cref{subsec:unsupervised-traditional-segmentation}, Random Decision
Forests are described in \cref{subsec:random-forests}, Markov Random Fields in
\cref{subsec:markov-random-fields} and \glspl{SVM} in
\cref{subsec:trad-SVM}.
Pre- and Postprocessing are covered in \cref{subsec:preprocessing-methods} and
\cref{subsec:post-processing-methods}. The respective advantages of the
classifiers are discussed in \cref{subsec:traditional-approaches-discussion}.

%!TEX root = vorlage.tex
% Martin Thoma

\subsection{Features}\label{subsec:features}%
The choice of features is very important in traditional approaches and a lot
of effort was done to find good features.

Poselets were used successfully in \cite{bourdev2010detecting,brox2011object}.
Those features rely on manually added extra keypoints such as \enquote{right
shoulder}, \enquote{left shoulder}, \enquote{right knee} and \enquote{left
knee}. They were originally used for human pose estimation. Finding those extra
keypoints is easily possible for well-known image classes like humans. However,
it is difficult for classes like airplanes, ships, organs or cells where the
human annotators do not know the keypoints. Additionally, the keypoints have to
be chosen for every single class. There are strategies to deal with those
problems like viewpoint-dependent keypoints.

Image edges and texture patches were proposed in~\cite{brox2011object}.

Other features include:

\begin{itemize}
    \item Pixel color in different image spaces (e.g. 3~features for RGB,
          3~features for HSV, 1~feature for the gray-value).
    \item TODO!
\end{itemize}

%!TEX root = vorlage.tex
% Martin Thoma

\subsection{Unsupervised Segmentation}%
\label{subsec:unsupervised-traditional-segmentation}%

Unsupervised segmentation algorithms can be used in supervised segmentation as
another source of information. While unsupervised segmentation algorithms can
never be semantic, they are well-studied and deserve at least a very brief
overview.

Semantic segmentation algorithms store information about the classes they were
trained to segment while non-semantic segmentation algorithms try to detect
consistent regions and region boundaries.

\subsubsection{Clustering Algorithms}
The $k$-means algorithm is a general-purpose clustering algorithm which
requires the number of clusters to be given beforehand. Initially, it places
the $k$ centroids randomly in the feature space. Then it assignes each
data point to the nearest centroid, moves the centroid to the center of the
cluster and continues the process until a stopping criterion is reached. A
faster variant is described in \cite{hartigan1975clustering}.

$k$-means was applied by~\cite{chen1998image} for medical image segmentation.

Another clustering algorithm is the mean-shift algorithm which was introduced
by~\cite{comaniciu2002mean} for segmentation tasks. The algorithm first applies
mean shift filtering on the original image data and then clusters the remaining
points.

% TODO: Understand this algorithm. Use \cite{comaniciu2002mean} and
% \cite{pantofaru2005comparison} for it.


\subsubsection{Graph Based Image Segmentation}%
\label{subsec:graph-based-image-segmentation}%
Graph-based image segmentation algorithms typically interpret pixels as
vertices and an edge weight is a measure of
dissimilarity such as the difference in color~\cite{felzenszwalb2004efficient}.

\href{http://cs.brown.edu/~pff/segment/}{cs.brown.edu}.

TODO: See also \cite{pantofaru2005comparison} for another description.

A graph-based method which got the 2nd rank in the Pascal VOC 2010
challenge~\cite{everingham2010pascal} is described
in~\cite{carreira2010constrained}.


\subsubsection{Random Walks}

Random walks belong to the graph-based image segmentation algorithms. Random
walk image segmentation usually works as follows: Some seed points are placed
on the image for the different objects in the image. From every single pixel,
the probability to reach the different seed points by a random walk is
calculated. This is done by taking image gradients as described in
\cref{subsec:features} for \gls{HOG} features. The class of the pixel is the
class of which a seed point will be reached with highest probability. At first,
this is an interactive segmentation method, but it can be extended to be
non-interactive by using another segmentation methods output as seed points.

It was shown in~\cite{meilpa2001learning} that normalized cuts
(NCuts)~\cite{shi2000normalized} can be expressed with random walks.


\subsubsection{Active Contour Models}

\Glspl{ACM} are segmentation models segment images roughly along edges, but
try also to minimize a so called \textit{energy function}. They were initially
described in~\cite{kass1988snakes}. \Glspl{ACM} be used to segment an image or
to refine segmentation as it was done in~\cite{atkins1998fully} for brain
\gls{MR} images.


\subsubsection{Watershed Segmentation}\label{subsec:watershed}
The watershed algorithm takes a grayscale image and interprets it as a height
map. Low values are catchment basins and the higher values between two
neighboring catchment basins is the watershed. The catchment basins should
contain what the developer wants to capture. This implies that those areas
must be dark on grayscale images. The algorithm starts to fill the basins from
the lowest point. When two basins get connected, a watershed is found. The
algorithm stops when the highest point is reached.

A detailed description of the watershed segmentation algorithm is given
in~\cite{roerdink2000watershed}.

The watershed segmentation was used in~\cite{1260033} to segment white blood
cells. As the authors describe, the segmentation by watershed transform has
two flaws: Over-segmentation due to local minima and thick watersheds due to
plateaus.
%!TEX root = vorlage.tex

\subsection{Random Decision Forests}\label{subsec:random-forests}

Random Decision Forests were first proposed in~\cite{ho1995random}. This type
of classifier applies techniques called \textit{ensemble learning}, where
multiple classifiers get trained and a combination of their hypotheses is
used. One ensemble learning technique is the \textit{random subspaces} method
where each classifier gets trained on a random subspace of the feature~space.
Another ensemble learning technique is \textit{bagging}, which is training the
trees on random subsets of the training~set. In the case of Random Decision
Forests, the classifiers are decision trees. A decision tree is a tree where
each inner node uses one or more features to decide in which branch to descend.
Each leaf is a class.

A strength of Random Decision Forests compared to many other classifiers like
\glspl{SVM} and neural networks is that the scale of measure of the features
(nominal, ordinal, interval, ratio) can be arbitrary.

Random decision trees were extensively studied in the past 20~years and a
multitude of training algorithms has been proposed (e.g. ID3
in~\cite{quinlan1986induction}, C4.5 in~\cite{quinlan2014c4}). Possible
training hyperparameters are the measure to evaluate the \enquote{goodness of
split}~\cite{raey89empirical}, the number of decision trees being used, and if
the depth of the trees is restricted. Typically in the context of
classification, random decision trees are trained by adding new nodes until
each leaf contains only nodes of a single class or until it is not possible to
split further. This is called a \textit{stopping criterion}.

There are two typical training modes: \textit{Central axis projection} and
\textit{perceptron training}. In training, for each node a hyperplane is
searched which is optimal according to an error function.

Random Decision Forests with texton features (see \cref{subsubsec:textons}) are
applied in~\cite{shotton2008semantic} for segmentation. In the~\cite{MSCR-db}
dataset, they report a per-pixel accuracy rate of \SI{66.9}{\percent} for their
best system. This system needs \SI{415}{\milli\second} for the segmentation of
$\SI{320}{\pixel} \times \SI{213}{\pixel}$ images on a single
\SI{2.7}{\giga\hertz} core. On the Pascal VOC~2007 dataset, they report an
average per-pixel accuracy for their best segmentation system
of~\SI{42}{\percent}.

%!TEX root = vorlage.tex
% Martin Thoma

\subsection{Markov Random Fields}\label{subsec:markov-random-fields}
% TODO: Probability space?
A \Gls{MRF} is a tupel $(G, X)$, where $G=(V,E)$ is an undirected graph and
$X=(X_V)_{v \in V}$ is a set of random variables which satisfy the pairwise
Markov property, the local Markov property and the global Markov property:

% https://en.wikipedia.org/wiki/Markov_random_field#Definition
% \begin{itemize}
%     \item \textbf{Pairwise Markov property}: $X_u \independent X_v | X_{V \setminus \Set{u, v}}$ if $\Set{u,v} \notin E$
%     \item \textbf{Local Markov property}: $X_u \independent X_{V \setminus \Set{v | v \text{is } u \text{ or a neighbor of } u}} | X_{\text{neighbors of } v}$
%     \item \textbf{Global Markov property}: $X_A \independent X_B | X_S$ where
%           every path from a node in $A$ to a node in $B$ passes through $S$
% \end{itemize}

% is a probabilisitic model. They have the Markov property which is
% typically displayed in an undirected graph which shows dependencies between the
% random variables.

\Glspl{CRF} and Boltzmann Machines are a variations of Markov random fields.

% http://www.mathunion.org/ICM/ICM1986.2/Main/icm1986.2.1496.1517.ocr.pdf
%
% Segmentation of brain MR images through a hidden Markov random field model
% and the expectation-maximization algorithm \cite{zhang2001segmentation}

% In short: specify locally and model globally.

% define only local properties. allows by transitivity to get a model
% for global properties.

% definition:

% A \gls{MRF} is a countable set of random variables. In the task of segmentation,
% they are often index-based spacial positions.

% \begin{itemize}
%     \item What is your set of random models? (e.g. Ising model: one for each pixel)
%           \begin{itemize}
%               \item Irving-Model: \cite{boykov2000interactive}
%               \item Potts-Model: \cite{boykov2001fast}
%           \end{itemize}
% \end{itemize}

% On those, define an undirected hypergraph $G(V, E)$. $V$ are pixels, $E$ is
% typically defined by neighboring pixels.

% good: 

% \begin{itemize}
%     \item Convenient modeling:Just write down energie function (in some cases)
%           do define entire model
%     \item fast inference (only 1 or 2 variants)
%     \item sometimes good performance
% \end{itemize}

% bad:

% \begin{itemize}
%     \item learning is difficult
% \end{itemize}



% ---


From \cite{yang2012layered}:

> In contrast, semantic segmentation models have largely been built on top of
Markov Random Field (MRF) models which enforce smoothness across pixel labels

\begin{itemize}
    \item X. He, R. Zemel, and M. Carreira-Perpinan, “Multiscale Conditional
          Random Fields for Image Labeling,” Proc. IEEE CS Conf. Computer
          Vision and Pattern Recognition, vol. 2, 2004.
    \item A. Torralba, K. Murphy, and W. Freeman, “Contextual Models for
          Object Detection Using Boosted Random Fields,” Proc. Advances in
          Neural Information Processing Systems, 2004.
    \item S. Kumar and M. Hebert, “A Hierarchical Field Framework for
          Unified Context-Based Classification,” Proc. 10th IEEE Int’l Conf.
          Computer Vision, vol. 2, 2005.
    \item \Glspl{CRF} were applied in \cite{shotton2006textonboost}.
    \item Z. Tu, “Auto-Context and Its Application to High-Level Vision
          Tasks,” Proc. IEEE Conf. Computer Vision and Pattern Recognition,
          2008.
\end{itemize}


A method similar to \glspl{CRF} was proposed in~\cite{gonfaus2010harmony}.
The system of Gonfaus~et.al. ranked~1st by mean accuracy in the segmentation
task of the PASCAL VOC 2010 challenge~\cite{everingham2010pascal}.

TODO:

\begin{itemize}
    \item "Associative hierarchical CRF" by "Oxford Brookes University", "Lubor Ladicky Christopher Russell Philip Torr", see \href{http://host.robots.ox.ac.uk/pascal/VOC/voc2010/results/index.html}{pascal voc 2010}
\end{itemize}
%!TEX root = vorlage.tex
% Martin Thoma

\subsection{SVMs}\label{subsec:trad-SVM}%

% An Introduction to Support Vector Machines and Other Kernel-based Learning Methods
% Support-Vector Networks by Vapnik
% Learning with kernels: Support vector machines, regularization, optimization, and beyond
\Glspl{SVM} are well-studied binary classifiers which can be described by five
central ideas. For those ideas, the training data is represented as
$(\textbf{x}_i, y_i)$ where $\textbf{x}_i$ is the feature vector and $y_i \in
\Set{-1, 1}$ the binary label for training example $i \in \Set{1, \dots, m}$.


\begin{enumerate}
    \item If data is linearly seperable, it can be seperated by a hyperplane.
          There is one hyperplane which maximizes the distance to the
          datapoints. This hyperplane should be taken:\\
          \begin{equation*}
          \begin{aligned}
              \min_{\textbf{w}, b}\,&\frac{1}{2} \|\textbf{w}\|^2\\
              \text{subject to }& \forall_{i=1}^m y_i \cdot \underbrace{(\langle \textbf{w}, \textbf{x}_i\rangle + b)}_{\mathclap{\sgn \text{ applied to this gives the classification}}} \geq 0
          \end{aligned}
          \end{equation*}

          When the space gets normalized one can also demand a margin by re-formulating
          the constraint to
          $\forall_{i=1}^m y_i \cdot (\langle \textbf{w}, \textbf{x}_i\rangle + b) \geq 1$.
    \item The primal problem is to find the normal vector $\textbf{w}$ and the
          bias $b$. The dual problem is to express $\textbf{w}$ as a linear
          combination of the training data $\textbf{x}_i$:
          \[\textbf{w} = \sum_{i=1}^m \alpha_i y_i \textbf{x}_i\]
          where $y_i \in \Set{-1, 1}$ represents the class of the training
          example.
    \item Not every dataset is linearly seperable. This problem is approached
          by transforming the dataset with a non-linear mapping into a higher
          dimensional space $\Phi$.
    \item Implicitly using a high-dimensional (probably $\infty$-dimensional)
          space. % TODO: http://www.jstor.org/stable/pdf/25464664.pdf?acceptTC=true
          % Schölkopf: Learning with kernels
          % Kernel-Trick
    \item Introduction of slack variables to relax the requirement of linear
          seperability. The trade-off between accepting some errors and a more
          complex model is weighted by a parameter $C \in \mathbb{R}_0^+$. The
          bigger $C$, the more errors are accepted. The new optimization
          problem is:
          \begin{equation*}
          \begin{aligned}
              \min_{\textbf{w}}\,&\frac{1}{2} \|\textbf{w}\|^2 + C \cdot \sum_{i} \xi_i\\
              \text{subject to }& \forall_{i=1}^m y_i \cdot (\langle \textbf{w}, \textbf{x}_i\rangle + b) \geq 1 - \xi_i
          \end{aligned}
          \end{equation*}
\end{enumerate}

The described \glspl{SVM} can only distinguish between two classes. Common
strategies to expand those binary classifiers to multi-class classification is
the \textit{one-vs-all} and the \textit{one-vs-one} strategy. In the one-vs-all
strategy $n$ classifiers have to be trained which can distingish one of the $n$
classes against all other classes. In the one-vs-one strategy $\frac{n^2 - n}{2}$
classifiers are trained; one classifier for each pair of classes.

A detailed description of \glspl{SVM} can be found in \cite{burges1998tutorial}.

\Glspl{SVM} are used by \cite{yang2012layered} on the 2009 and 2010 PASCAL
segmentation challenge~\cite{everingham2010pascal}. They did not hand their
classifier in to the challenge itself, but calculated an average rank~of~7
amongst the different categories.

\cite{felzenszwalb2010object} also used an SVM based method and achieved the
\nth{7}~rank in the 2010 PASCAL segmentation callenge by mean accuracy.


\subsection{Pre-processing methods}\label{subsec:preprocessing-methods}%
A typical image is opened in RGB color space, but depending on the classifier
and the problem another color space might result in better segmentations. RGB,
YcBcr, HSL, Lab and YIQ are some examples used by \cite{cohen2015memory}.

\Gls{PCA} is applied by~\cite{chen2011pixel} to reduce the dimensionality of
the feature space.

\subsection{Post-processing methods}%
\label{subsec:post-processing-methods}%
Post-processing methods are an important part of traditional pixel-level
segmentation approaches. Active contour models are one example of a
post-processing method~\cite{kass1988snakes}.

A refinement of the found segmentation can be obtained by adjusting a found
semantic segmentation to match close edges. This was used
in~\cite{brox2011object} with an ultra-metric contour map
Map~\cite{arbelaez2009contours}


\subsection{Discussion}%
\label{subsec:traditional-approaches-discussion}%
According to \cite{pantofaru2005comparison}, the mean shift algorithm produces
segmentations that correspond well to human perception, but it is sensitive to
its parameters. Depending on them, different granularities of the segmentation
can be achieved. \cite{pantofaru2005comparison} draws the conclusion, that
the segmentations found by the graph based image segmentation approach
in \cite{felzenszwalb2004efficient} are inferior to the segmentations found
by the mean shift algorithm described in \cite{comaniciu2002mean}.
