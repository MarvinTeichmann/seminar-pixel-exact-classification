%!TEX root = vorlage.tex
% Martin Thoma

\subsection{Features}\label{subsec:features}%
The choice of features is very important in traditional approaches.

\textit{Poselets} were used successfully in \cite{bourdev2010detecting,brox2011object}.
Those features rely on manually added extra keypoints such as \enquote{right
shoulder}, \enquote{left shoulder}, \enquote{right knee} and \enquote{left
knee}. They were originally used for human pose estimation. Finding those extra
keypoints is easily possible for well-known image classes like humans. However,
it is difficult for classes like airplanes, ships, organs or cells where the
human annotators do not know the keypoints. Additionally, the keypoints have to
be chosen for every single class. There are strategies to deal with those
problems like viewpoint-dependent keypoints.

Image edges and texture patches were proposed in~\cite{brox2011object}.

Other features include:

\begin{itemize}
    \item Pixel color in different image spaces (e.g. 3~features for RGB,
          3~features for HSV, 1~feature for the gray-value).
    \item \Gls{HOG} features interpret the image as a discrete function
          $I: \mathbb{N}^2 \rightarrow \Set{0, \dots, 255}$ which maps the
          position to a color. For each pixel, there are two gradients: The
          partial derivative of $x$ and $y$. Now the original image got
          transformed to two feature maps of equal size which represents the
          gradient. These feature maps are splitted into patches and a
          histogram of the directions is calcualted for each patch. \gls{HOG}
          were proposed in~\cite{1467360} and
          are used in~\cite{bourdev2010detecting,felzenszwalb2010object}.
    \item \Gls{SIFT} feature descriptors find keypoints in an image. The image patch
          of the size $16 \times 16$ around the keypoint is taken. This patch
          is divided in $16$ distinct parts of the size $4 \times 4$. For each
          of those parts a histogram of 8~orientations is calculated similar as
          for \gls{HOG} features. This results in a 128-dimensional feature
          vector for each keypoint.

          \Gls{SIFT} is described in detail in~\cite{raey}.

          % SIFT descriptors are multi-image representations of an image
          % neighborhood. They are Gaussian derivatives computed at 8
          % orientation planes over a 4x4 grid of spatial locations, giving a
          % 128-dimension vector. Fig. 1 shows an example of the maps of
          % gradient magnitude corresponding to the 8 orientations.
    \item \Gls{BOV}, also called \textit{bag of keypoints}, is based on vector
          quantization. Similar to \gls{HOG} features, \gls{BOV} features are
          histograms which count the number of occurences of certain patterns
          within a patch of the image. \Gls{BOV} are described
          in~\cite{csurka2004visual} and used in combination with \gls{SIFT}
          feature descriptors in~\cite{csurka2008simple}.
\end{itemize}
