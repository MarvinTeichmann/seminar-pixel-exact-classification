%!TEX root = vorlage.tex

\subsection{Features and Preprocessing methods}\label{subsec:features}%
The choice of features is very important in traditional approaches.
The most commonly used features are explained in the following.

\subsubsection{Pixel Color}
Pixel color in different image spaces (e.g. 3~features for RGB, 3~features for
HSV, 1~feature for the gray-value). A typical image is in the RGB color space,
but depending on the classifier and the problem another color space might
result in better segmentations. RGB, YcBcr, HSL, Lab and YIQ are some examples
used by \cite{cohen2015memory}. No single color space has been proven to be
superior to all others in all contexts~\cite{cheng2001color}. However, the most
common choices seem to be RGB and HSI. Reasons for choosing RGB is simplicity
and the support by programming languages, whereas the choice of the HSI color
space might make it simpler for the classifier to get invariant to
illumination. One reason for choosing CIE-L*a*b* color space is that it
approximates human perception of brightness~\cite{kasson1992analysis}. It
follows that choosing the L*a*b color space helps algorithms to detect
structures which are seen by humans. Another way of improving the structure
within an image is histogram equalization, which can be applied to improve
contrast~\cite{pizer1987adaptive,4228537}.

\subsubsection{Histogram of oriented Gradients}
\Gls{HOG} features interpret the image as a discrete function
$I: \mathbb{N}^2 \rightarrow \Set{0, \dots, 255}$ which maps the position $(x,
y)$ to a color. For each pixel, there are two gradients: The partial derivative
of $x$ and $y$. Now the original image got transformed to two feature maps of
equal size which represents the gradient. These feature maps are splitted into
patches and a histogram of the directions is calculated for each patch.
\gls{HOG} features were proposed in~\cite{1467360} and are used
in~\cite{bourdev2010detecting,felzenszwalb2010object}.


\subsubsection{SIFT}
\Gls{SIFT} feature descriptors find keypoints in an image. The image patch of
the size $16 \times 16$ around the keypoint is taken. This patch is divided in
$16$ distinct parts of the size $4 \times 4$. For each of those parts a
histogram of 8~orientations is calculated similar as for \gls{HOG} features.
This results in a 128-dimensional feature vector for each keypoint.

\Gls{SIFT} is described in detail in~\cite{raey}.


\subsubsection{BOV}
\Gls{BOV}, also called \textit{bag of keypoints}, is based on vector
quantization. Similar to \gls{HOG} features, \gls{BOV} features are histograms
which count the number of occurrences of certain patterns within a patch of the
image. \Gls{BOV} are described in~\cite{csurka2004visual} and used in
combination with \gls{SIFT} feature descriptors in~\cite{csurka2008simple}.


\subsubsection{Poselets}
\textit{Poselets} were used successfully
in~\cite{bourdev2010detecting,brox2011object}. Those features rely on manually
added extra keypoints such as \enquote{right shoulder}, \enquote{left
shoulder}, \enquote{right knee} and \enquote{left knee}. They were originally
used for human pose estimation. Finding those extra keypoints is easily
possible for well-known image classes like humans. However, it is difficult for
classes like airplanes, ships, organs or cells where the human annotators do
not know the keypoints. Additionally, the keypoints have to be chosen for every
single class. There are strategies to deal with those problems like
viewpoint-dependent keypoints.


\subsubsection{Textons}\label{subsubsec:textons}
A \textit{texton} is the minimal building block of vision. The computer vision
literature does not give a strict definition for textons, but edge detectors
could be one example. One might argue that deep learning techniques with
\glspl{CNN} learn textons in the first filters.

An excellent explanation of textons can be found in~\cite{zhu2005textons}.


\subsubsection{Dimensionality Reduction}
High-resolution images have a lot of pixels. Having one or more feature per
pixel results in many features. This makes training difficult while the higher
resolution might not contain much more information. A simple approach to deal
with this is downsampling the high-resolution image to a low-resolution
variant. Another way of doing dimensionality reduction is \gls{PCA}, which is
applied by~\cite{chen2011pixel}. The idea behind \gls{PCA} is to find a
hyperplane on which all feature vectors can be projected with a minimal loss of
information. A detailed description of \gls{PCA} is given
by~\cite{smith2002tutorial}.

One problem of \gls{PCA} is the fact that it does not distinguish different
classes. This means it can happen that a perfectly linearly separable set of
feature vectors gets not separable at all after applying \gls{PCA}.

There are many other techniques for dimensionality reduction. An overview and
a comparison over some of them is given by~\cite{van2009dimensionality}.
