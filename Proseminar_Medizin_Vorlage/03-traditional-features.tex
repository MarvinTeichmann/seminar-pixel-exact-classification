%!TEX root = vorlage.tex
% Martin Thoma

\subsection{Features}\label{subsec:features}%
The choice of features is very important in traditional approaches and a lot
of effort was done to find good features.

Poselets were used successfully in \cite{bourdev2010detecting,brox2011object}.
Those features rely on manually added extra keypoints such as \enquote{right
shoulder}, \enquote{left shoulder}, \enquote{right knee} and \enquote{left
knee}. They were originally used for human pose estimation. Finding those extra
keypoints is easily possible for well-known image classes like humans. However,
it is difficult for classes like airplanes, ships, organs or cells where the
human annotators do not know the keypoints. Additionally, the keypoints have to
be chosen for every single class. There are strategies to deal with those
problems like viewpoint-dependent keypoints.

Image edges and texture patches were proposed in~\cite{brox2011object}.

Other features include:

\begin{itemize}
    \item Pixel color in different image spaces (e.g. 3~features for RGB,
          3~features for HSV, 1~feature for the gray-value).
    \item TODO!
\end{itemize}
