%!TEX root = vorlage.tex
% Martin Thoma

\subsection{Unsupervised Segmentation}%
\label{subsec:unsupervised-traditional-segmentation}%

Unsupervised segmentation algorithms can be used in supervised segmentation as
another source of information. While unsupervised segmentation algorithms can
never be semantic, they are well-studied and deserve at least a very brief
overview.

Semantic segmentation algorithms store information about the classes they were
trained to segment while non-semantic segmentation algorithms try to detect
consistent regions and region boundaries.

\subsubsection{General-Purpose Clustering Algorithms}
% The mean shift algorithm was introduced by~\cite{comaniciu2002mean} for
% segmentation tasks. The algorithm first applies mean shift filtering on the
% original image data and then clusters the remaining points.
%
% TODO: Understand this algorithm. Use \cite{comaniciu2002mean} and
% \cite{pantofaru2005comparison} for it.

% \subsubsubsection{$k$-Means}\label{subsec:k-means}

$k$-means~\cite{hartigan1975clustering},
mean shift clustering~\cite{comaniciu2002mean}

\subsubsection{Graph Based Image Segmentation}%
\label{subsec:graph-based-image-segmentation}%
\href{http://cs.brown.edu/~pff/segment/}{cs.brown.edu}
See \cite{felzenszwalb2004efficient}.

TODO: See also \cite{pantofaru2005comparison} for another description.

\subsubsection{Random Walks}
Notes from \cite{meilpa2001learning}:

\begin{itemize}
    \item Normalized Cut (see \cite{shi2000normalized}) seems to be a special
          case of this
    \item Seems to be a \enquote{spectral method}
    \item This is pairwise clustering. Clusters points which are similar and
          optimize e.g. for maximum total intracluster similarity. In contrast,
          to statistical clustering which assumes a probabilistic model which
          generates the data, this only defines a similarity function.
    \item Spectral clustering is a similarity based method. Spectral clustering
          methods use eingenvalues / eigenvectors of the matrix obtained by the
          similarity function.
\end{itemize}


\subsubsection{Edge Detection}

See \cite{kass1988snakes}.


\subsubsection{Watershed Segmentation}\label{subsec:watershed}
The watershed algorithm takes a grayscale image and interprets it as a height
map. Low values are catchment basins and the higher values between two
neighboring catchment basins is the watershed. The catchment basins should
contain what the developer wants to capture. This implies that those areas
must be dark on grayscale images. The algorithm starts to fill the basins from
the lowest point. When two basins get connected, a watershed is found. The
algorithm stops when the highest point is reached.

A detailed description of the watershed segmentation algorithm is given
in~\cite{roerdink2000watershed}.

The watershed segmentation was used in~\cite{1260033} to segment white blood
cells.



% \subsubsubsection{Watershed Algorithm}
% \begin{itemize}
%     \item Apply to image intensity gives some long superpixels
%     \item apply to image gradient magnitude gives rounder superpixels
% \end{itemize}