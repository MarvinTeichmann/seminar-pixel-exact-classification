%!TEX root = vorlage.tex
% Marvin Teichmann and Martin Thoma
\section{Taxonomy of Segmentation Algorithms}\label{sec:taxonomy}
Taking a step back, the task of semantic segmentation can be grouped in several
categories:

\begin{itemize}
    \item \textbf{By possible classes}: Are the classes which should be distinguised
          known at the time when the algorithm is developed / trained or might
          there occur several objects which were never seen before?
          \begin{itemize}
              \item Fixed-class: All classes are known at training time.
                    \begin{itemize}
                        \item Binary: street/no street.
                    \end{itemize}
              \item Open-class: There might be completely new classes
          \end{itemize}
    \item \textbf{By class affiliation of pixels}:
          \begin{itemize}
              \item Single class affiliation: Every pixel belongs to exactly one class. There might
                    be probabilities, but a pixel cannot belong with a 100\%
                    probability to two classes.
              \item Multiple class affiliation: A single pixel might belong to
                    multiple classes. An example is a glass on a table: One
                    knows that the glass is there and it is possible to see the
                    table below / behind. Another example is hair. Mattening
                    methods produce maps which reflect this property for hair.\cite{levin2008spectral}
          \end{itemize}
    \item \textbf{By input data}:
          \begin{itemize}
              \item greyscale / colored / with depth (RGB-D)
              \item single image (NOTE: we write about this) / time series (NOTE: only mention it)
              \item pixels / voxels \cite{wolz2012multi}
          \end{itemize}
    \item \textbf{Operation state}:
          \begin{itemize}
              \item completely automatically <-- we write about this type
              \item interactive (e.g. user clicks on background or user makes a
                    coarse segmentation and automatically refinement) as
                    described in
                    \cite{protiere2007interactive,rother2004grabcut}.
              \item active as in
                    \cite{schiebener2011segmentation,schiebener2012discovery} or
                    passive, where the received image cannot be influenced
          \end{itemize}
\end{itemize}
